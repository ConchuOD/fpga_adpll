\section{Conclusions}
%TODO dont reset acro
The objective of this project was to create \iacs{FPGA} based platform that could be used to prototype and model \acs{ADPLL} networks in a cost and time effective manner. The main effort of the project was to analyse the ways in which the components of \iacs{ADPLL} network could be designed, then select from this pool a number of the most appropriate designs. A number of \acs{ADPLL} designs were then created using the selected components in different combinations for the purposes of comparison. Once these \acs{ADPLL} designs had been validated, they were implemented in networks of a 2x2 and 3x3 Cartesian grid topology. Measurements and performance benchmarks were then carried out on these \acs{ADPLL} networks, and the results used to verify claims made regarding the intended use cases of the individual \acs{ADPLL} designs. Finally various small modifications to the \acs{ADPLL} designs were made, and their impact on system performance identified.

The project resulted in a platform comprising three \acs{ADPLL} designs suited for different use cases. As evidenced by the investigation into the impact of minor variations, this system fulfils the objective that it be suitable for the testing of architectural changes. Realistic, though exaggerated, variation due to implementation makes the inverter based designs ideal for the testing of new control schemes or performance enhancing modules. Simultaneously, this makes designs based on \acsp{RO}   ideal for the confirmation of either theoretical or simulated results, in a platform that suffers from realistic implementation based variation, and non-ideal effects such as power supply noise that are difficult to simulate well. The presence of \acs{FPGA} clocked \acsp{DCO} and \acsp{PFD} makes the emulation of mixed-signal circuits at a scaled down frequency for the purposes of design verification possible.

As the platform is designed with extensibility in mind, rapid changes in the design are possible, with the lengthy and resource consuming tape-out process that would otherwise be required for a test in hardware, cut down to the time taken for the \ac{EDA} tool of choice to generate a configuration file. Whereas tape-out may be a costly process, for a figure in the $100$s of Euro, an \acs{FPGA} can be acquired will that bridge the gap between software simulations and implementation in custom silicon, by providing a hardware-based simulation, modelling and validation platform.

\section{Suggested Future Work}
\subsection{\acl{TDL} Characterisation}
\acreset{TDL}
As previously mentioned, characterisation of the \ac{TDL} was not carried out due to the time it would have taken to perform the required measurements and tests. This characteristion can be accomplished through the use of an off-chip tunable delay. Off-chip is unfortunately a requirement, as the only delays available of the order required to test the step sizes are the variable delay of the inverters themselves. A resistor \& capacitor delay circuit can be used to perform these test, and the output of the \ac{PFD} analysed on a mixed-signal oscilloscope alongside the delayed and reference waveforms. The use of an off-chip delay element necessitates careful construction/cailbration of the measurement environment to eliminate the delay stemming from the signal path through input/output buffers as it leaves the chip.

\subsection{Increased Network Size}
A 4x4 network was a stretch-goal of this project, depending on the time remaining after once the minumim goal of a 3x3 network had been reached. Unfortuantely due to problems with the clocked phase detector which resulted in mode-locking, incorrect \ac{RO} inverter selection logic and the projected time requirement to ensure alignment of each \ac{RO} this was not achieved. A 4x4 network would better explain some dynamics, such as the skew growth as the \ac{ADPLL} indices increase. However. progressing beyond a 3x3 network means that less than $20\%$ of \ac{ADPLL} output signals can be examined at any given point in time. As such, it may be useful to develop a more advanced measurement set-up that can either capture more signals simultaneously, or implement a measurement system that makes more efficient use of the available probes.

\subsection{\acs{FPGA} Clocked, Linear Period \acs{DCO}}
\acreset{FPGA,DCO}
While this design of \ac{DCO} was proposed earlier in this thesis, it was not implemented in any \ac{ADPLL}s, as it was only realised that the output waveform did not matter, just the location of the rising edges was not made until later in the project, once the pitfall encountered in the clocked \ac{PFD} design was resolved. Until this point the non-square wave output had eliminated the design from consideration. However, as in \iac{ASIC} linear period designs are easier to implement, it would be ideal to have the benefit of frequency scaled hardware simulations with \iac{DCO} that more closely relates the the eventual system.

\subsection{Procedural Network Instantiation}
This suggestion is directed at making the user experience simpler. As implemented, networks based on \ac{ADPLL}s containing \ac{RO}s are implemented in separate files to their \ac{FPGA} clocked \ac{DCO} based counterparts as are the networks of difference sizes. At a network size of 2x2 this was not a problem, but at the scale of 3x3 network the modules became somewhat cumbersome to work with, with a large number of interconnects. For 3x3 and larger networks it would be expedient to create the network with a \texttt{generate} statement, as this would reduce the number of lines of code and the amount of named interconnects required. Instead the user could set a number of configuration parameters and the network would be created accordingly. This could even allow for the use of different \ac{ADPLL} types in the same network with ease. Unfortunately the tasks of floorplanning and oscillator alignment would still be left to the user to perform.
