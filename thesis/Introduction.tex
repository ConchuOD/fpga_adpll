This thesis will put forward the \ac{FPGA} as a tool in the design of \ac{ADPLL} networks to bridge the gap between software simulations and implementation in custom silicon by providing a hardware-based simulation, modelling and validation platform. While an FPGA lacks the direct control over the schematic and layout that an \ac{ASIC} provides, the hardware nature of this platform enables the analysis of system dynamics that are not easily modelled in simulation and the testing. \ac{ADPLL}s are an entirely digital implementation of a \ac{PLL}, advantageous as this allows for easier integration into the modern high transistor count \ac{IC}s for which \ac{ADPLL} networks are a proposed clock distribution system.

The goal of this project is to design and implement an extensible platform that can be used by the research team in \ac{UCD} going forward as they seek to understand the behaviour of \ac{ADPLL} networks at a higher level and to serve as a hardware test-bed for proposed new architectures or system components. In order to accomplish this goal a number of potential \ac{ADPLL} architectures will be investigated, implemented and tested to ensure they are function correctly. Individual \acsp{ADPLL}, however, will not give sufficient insight into the behaviour of a network, so once the \ac{ADPLL} designs have been established, each will be implemented as part of a network of increasing sizes. Each contrasting design will be analysed based on the results of measurements and tests, and these results will be used to corroborate claims made regarding which \ac{FPGA} based \ac{ADPLL} design or \ac{ADPLL} network architecture is best suited for particular use cases.

The design of each block, or component part, used in the network will be discussed, starting with the reasons for their selection and an explanation of the design methodology, along with any major pitfalls encountered their creation. The impact on performance caused by modifying the design of these blocks will again be assessed on the basis of measurement results, before comparison is made to both theoretical expectations and their use case.

An \ac{FPGA} based test platform is ideal for those who wish to examine system dynamics without the time delays, financial burden or expertise required to develop a complete mixed-signal system on an \ac{ASIC} but retain the ability to realistically simulate the behaviour of an \ac{ADPLL} network. The result of this project is such a platform, designed to be extensible, with flexibility built into each component/module used.