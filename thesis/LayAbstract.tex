\documentclass[11pt,english,british]{report}
\usepackage[a4paper]{geometry}
\geometry{verbose,tmargin=3cm,bmargin=3cm,lmargin=3.5cm,rmargin=2.4cm}
\begin{document}
With the increasing proliferation of ``smart'' devices the need for low power yet high speed devices has never been greater. Each smart device contains a processor to control the device, each containing complex circuitry, requiring extremely exact synchronisation. The task of this synchronisation falls to the ``clock'' signal which must occur at the same instant in all areas of the processor. Conventional solutions to this problem cannot satisfy both power and speed requirements simultaneously, so designers have proposed the ADPLL Network which approaches the problem from the other side. Rather than generate the clock once and send it around the chip, which consumes a large amount of power, an ADPLL network divides the chip into a grid and generates many signals that each serve a local area, only requiring non time critical control signals to be sent over large distances.\\
The production of chips to test designs is both expensive and time consuming so designers must ensure that mistakes have not been made, accomplished by simulating the behaviour of the designs using complex behavioural and theoretical models. This thesis will discuss the use of Field Programmable Gate Arrays (FPGA) as a hardware testing, simulation and validation platform, to be used prior to test chip production, that is both cost and time effective. The hardware nature of an FPGA enables the analysis of behaviours that would not be possible in software, without the cost and time penalties of a custom chip. This is made possible by the ability to reconfigure the chip at will, albeit with comparatively lesser capabilities.\\
This thesis implements networks using three designs of ADPLL, each using a different architecture, and highlights the use cases to which each is best suited. The performance of each design is then analysed and this compared to the suggested use cases. Additionally the impact of more minor architectural modifications is tested and documented.
%what do I do?
\end{document}