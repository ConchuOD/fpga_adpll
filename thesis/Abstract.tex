\documentclass[11pt,english,british]{report}
\usepackage[a4paper]{geometry}
\geometry{verbose,tmargin=3cm,bmargin=3cm,lmargin=3.5cm,rmargin=2.4cm}
\begin{document}
Low power, high frequency clock distribution systems will be in ever increasing demand in the near future as the need for high performance digital circuitry grows. At these frequencies, however, the conventional clock distribution systems are unable provide a clock signal of adequate quality without compromising on either of these problems. Many devices have turned away from using Globally Synchronous clock distribution systems in favour of those that divide the area of the chip into Globally Asynchronous but Locally Synchronous areas. However, new technologies seek to enable the use of Globally Synchronous methods at high frequencies, one of which being the use of oscillators coupled in phase, each responsible for delivering the clock to a subregion of the chip.\\
Each oscillator forms part of a Phase Lock Loop (PLL), and in order to enable the synchronisation of the network each PLL is linked those controlling the adjacent clock regions. For a digital system it is expedient to implement these PLLs digitally as an All-Digital Phase Lock Loop (ADPLL) as this provides a number of advantages. It has been shown in theory and experimentally that this method can produce the high quality clock desired with low power consumption.\\
As the production of test chips is expensive and time consuming through simulation and validation of a design is vital, traditionally carried out for mixed signal circuitry using complex behavioural and theoretical models. For a ADPLL Network the consistency of a signal both with respect to itself, and to the other clock signals on the chip is a key performance attribute and much of the variation is due to random processes that may be difficult to simulate effectively.\\
This thesis will demonstrate that the Field Programmable Gate Array (FPGA) can be used in order to simulate, model or validate ADPLL network architectures in a cost and time effective manner, as a complement to conventional methods. The key benefit is that many system dynamics that will be seen when a network is implemented on an Application Specific Integrated Circuit (ASIC), but be overlooked in software simulations can be examined in the hardware simulation that an FPGA provides.\\
This thesis implements networks using three designs of ADPLL, each using a different architecture, and highlights the use cases to which each is best suited. The performance of each design is then analysed and this compared to the suggested use cases. Additionally the impact of more minor architectural modifications is tested and documented.
\end{document}