\documentclass{beamer}
\mode<presentation>
{
  \usetheme{Warsaw}
  % or ...

  \setbeamercovered{transparent}
  % or whatever (possibly just delete it)
}

\title{ADPLL Network on an FPGA}
\author{Conor Dooley}
\subtitle{Semester 1 Presentation}

\begin{document}

\begin{frame}
    \titlepage
\end{frame}

\section*{Background Information}
\begin{frame}{Motivation}
  % - A title should summarize the slide in an understandable fashion
  %   for anyone how does not follow everything on the slide itself.

    \begin{itemize}
        \item
            Clock source for System-On-Chip devices.
        \item
            Goal of clocking system:\\
            Deliver in-phase clock signal to all parts of chip.
        \item
            Aim is to avoid synchronisation issues.
        \item
            Why are sync. errors bad?
        \begin{itemize}
            \item
                Forced to do less complex operations per clock cycle.
            \item
                $ \therefore $ must lower clock freq. or reduce complexity per cycle.
        \end{itemize}
    \end{itemize}
 
\end{frame}
\begin{frame}{Existing Solutions}
  % - A title should summarize the slide in an understandable fashion
  %   for anyone how does not follow everything on the slide itself.

    \begin{itemize}
        \item
            Solutions exist for this problem.
        \item
            Current implemenations have excessive power use/delays.
        \item
            Leads to skew/jitter at high frequency. %TODO what leads to this?
        \item
            Branch, H, X trees.
        \item
            Use buffers to distribute in sync clock.
        \item
            These suffer from fabrication mismatch problems.
        \item
            Solution to mismatches further drive up power cost.
    \end{itemize}
 
\end{frame}


\begin{frame}{Alternatives}
  % - A title should summarize the slide in an understandable fashion
  %   for anyone how does not follow everything on the slide itself.

    \begin{itemize}
        \item
            Attempts to fix mismatch: centralised/decentralised skew compensation.
        %TODO subitemize
        \item
            Centralised tries to via main oscillator tuning.
        \item
            Decentralised via tuning of paths %TODO this true?
        \item
            Both methods suffer from high power consumption.
        \item
            PLL network:
            PLLs generate clock in an area of chip
            synced via lower freq. error signal between blocks
            Laid out in carthesian grid
        \end{itemize}
 
\end{frame}
\section*{ADPLL Network}

\begin{frame}{ADPLL Network}
  % - A title should summarize the slide in an understandable fashion
  %   for anyone how does not follow everything on the slide itself.

    \begin{itemize}
        \item
            \textbf{Reminder:} A Phase Lock Loop is a circuit which outputs a signal phase synchronised with a multiple of a reference.
        \item
            ADPLL:
            %TODO subitem
        \item
            PLL using only digital blocks.
        \item
            Uses a number controller oscillator i.e. quantised frequency steps
        \item
            Digital loop filter.
        \item
            Quantised phase detector output, \"deadzone\" for phase detection
    \end{itemize}
\end{frame}

\begin{frame}{Network on an FPGA}
  % - A title should summarize the slide in an understandable fashion
  %   for anyone how does not follow everything on the slide itself.

    \begin{itemize}
        \item
            Why use this?
    \end{itemize}
 
\end{frame}

\begin{frame}{My ADPLL}
  % - A title should summarize the slide in an understandable fashion
  %   for anyone how does not follow everything on the slide itself.

     
\end{frame}

\section*{Measurements}
\begin{frame}{Measurements}
  % - A title should summarize the slide in an understandable fashion
  %   for anyone how does not follow everything on the slide itself.

    \begin{itemize}
        \item
            Why use this?
    \end{itemize}
 
\end{frame}

\section*{Future Work}

\begin{frame}{Future Work}
  % - A title should summarize the slide in an understandable fashion
  %   for anyone how does not follow everything on the slide itself.

    \begin{itemize}
        \item
            Why use this?
    \end{itemize}
 
\end{frame}

\section*{Summary}

\begin{frame}{Summary}

  % Keep the summary *very short*.
  \begin{itemize}
  \item
    The \alert{first main message} of your talk in one or two lines.
  \item
    The \alert{second main message} of your talk in one or two lines.
  \item
    Perhaps a \alert{third message}, but not more than that.
  \end{itemize}

  % The following outlook is optional.
  \vskip0pt plus.5fill
  \begin{itemize}
  \item
    Outlook
    \begin{itemize}
    \item
      Something you haven't solved.
    \item
      Something else you haven't solved.
    \end{itemize}
  \end{itemize}
\end{frame}



% All of the following is optional and typically not needed.
\appendix
\section<presentation>*{\appendixname}
\subsection<presentation>*{For Further Reading}

\begin{frame}[allowframebreaks]
  \frametitle<presentation>{For Further Reading}

  \begin{thebibliography}{10}

  \beamertemplatebookbibitems
  % Start with overview books.

  \bibitem{Author1990}
    A.~Author.
    \newblock {\em Handbook of Everything}.
    \newblock Some Press, 1990.


  \beamertemplatearticlebibitems
  % Followed by interesting articles. Keep the list short.

  \bibitem{Someone2000}
    S.~Someone.
    \newblock On this and that.
    \newblock {\em Journal of This and That}, 2(1):50--100,
    2000.
  \end{thebibliography}
\end{frame}


\end{document}
