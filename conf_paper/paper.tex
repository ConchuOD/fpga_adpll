

\section*{Motivation}
The technical motivation underlying this paper is the creation of an Field Programmable Gate Arary (FPGA) based prototyping platform for All Digital Phase Lock Loop (ADPLL) networks. As the creation of Application Specific Integrated Circuits (ASICs) is an expensive and time consuming process, any mistake has the potential to be rather costly. As such prior to the manufacture of an IC it is important to ensure that any errors made in the design have been weeded out prior to this stage.
Simulations at either at theoretical or gate/transistor levels have global usage in minimising such errors due to the ubiquity of simulators and their ease of use.
However simulations are only as good as the model used to describe the dynamics of the system, and emulating real jitter and other behaviours of a complex system is a rather difficult task. %TODO wording of this last sentence
An FPGA based prototype allows the system designers to validate the performance of both design and simulation, particularly the response to key noise sources such as power supply noise. An FPGA is ideal for this task as it leverages the existing skillset of a digital designer, permits the re-use of certain blocks, and most importantly enables cost effective and rapid reworking of the design.\\
This paper will examine the process of creating such a system, highlight the differences between potential designs and address some of the challenges and pitfalls that may be encountered along the way. This paper will also demonstrate that FPGA based prototyping can play a central role on the pathway to the implementation of an ADPLL network on a custom chip, or indeed any number of similar applications. \