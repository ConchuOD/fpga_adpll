\documentclass[a4paper,12pt]{article}

\title{Conference Paper Critique}
\author{Conor Dooley}

\begin{document}

\maketitle

The intention is to submit this paper to the 2019 edition of the International Conference on Synthesis, Modelling, Analysis and Simulation Methods and Applications to Circuit Design (SMACD), a conference dedicated to the modelling and simulation of, among other areas, mixed-signal circuits which aligns with the paper's goal of modelling an ADPLL network, itself a mixed-signal circuit. The final deadline for submission to this conference is the 1st of April 2019, the Monday after this submission is due internally, which is quite convenient. The paper is most suited to this conference as it describes the challenges and realities of implementing a platform for the hardware modelling and simulation of a complex system in a cost and time inexpensive manner. The paper could also possibly be submitted to the 2019 3rd International Conference on Circuits, System and Simulation (ICCSS 2019) with a deadline of the 12th of April, however, this conference is in China so if this paper was to be accepted this would present a logistical difficulty.\\

This paper was inspired by the usage of FPGAs as a prototyping tool by Eldar Zianbetov and Chuan Shan in the course of their development of an ADPLL network on an ASIC, however this was carried out at a significantly reduced frequency of operation. Here in UCD the technique has been used by Eugene Koskin and Pierre Bisleaux in their as of yet unpublished paper to validate theoretical models developed by Eugene during the course of his PhD. The initial goal of the project was to investigate a number of possible architectures for this prototyping platform, and this paper explains the designs analysed and the benefits of each technique \\

The two most glaring omissions in the paper are the third oscillator design mentioned in the course of the paper yet is not analysed and the rather poor performance of Design ``2a'' (ring oscillator and clocked phase detector) which was unable to acquire lock at 5 MHz and had to be analysed at the reduced frequency of 1.25 MHz (the output of the in-circuit divider). %TODO
The third oscillator, briefly mentioned in the paper, is of the type most similar to that used by Zianbetlov/Shan in their paper, which like Oscillator 1 is made up of a counter driven by the FPGA clock. However, unlike Oscillator 1 it is linear in period rather than frequency thus making it more comparable to the ring oscillator design I used. It would make for a more appropriate comparison of designs and has greater correlation with real world usage had it been used but was not as Oscillator 1 was chosen as the clocked design early in the project.\\

There is also a lack of detail regarding the implementation of each module used in the design, beyond images of the detector and high level descriptions due to the limited space available. Similarly the technical motivation does not contain a particularly large number of references beyond those demonstrating the usage of FPGAs in this role and the explanation of where the idea of ADPLL networks came from. This is due to the lack of publications addressing hardware simulation of such networks and the assumption the reader will be aware of the blocks involved in a Phase Lock Loop.




\end{document}
